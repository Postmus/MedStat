% Options for packages loaded elsewhere
\PassOptionsToPackage{unicode}{hyperref}
\PassOptionsToPackage{hyphens}{url}
\PassOptionsToPackage{dvipsnames,svgnames,x11names}{xcolor}
%
\documentclass[
  letterpaper,
  DIV=11,
  numbers=noendperiod]{scrartcl}

\usepackage{amsmath,amssymb}
\usepackage{iftex}
\ifPDFTeX
  \usepackage[T1]{fontenc}
  \usepackage[utf8]{inputenc}
  \usepackage{textcomp} % provide euro and other symbols
\else % if luatex or xetex
  \usepackage{unicode-math}
  \defaultfontfeatures{Scale=MatchLowercase}
  \defaultfontfeatures[\rmfamily]{Ligatures=TeX,Scale=1}
\fi
\usepackage{lmodern}
\ifPDFTeX\else  
    % xetex/luatex font selection
\fi
% Use upquote if available, for straight quotes in verbatim environments
\IfFileExists{upquote.sty}{\usepackage{upquote}}{}
\IfFileExists{microtype.sty}{% use microtype if available
  \usepackage[]{microtype}
  \UseMicrotypeSet[protrusion]{basicmath} % disable protrusion for tt fonts
}{}
\makeatletter
\@ifundefined{KOMAClassName}{% if non-KOMA class
  \IfFileExists{parskip.sty}{%
    \usepackage{parskip}
  }{% else
    \setlength{\parindent}{0pt}
    \setlength{\parskip}{6pt plus 2pt minus 1pt}}
}{% if KOMA class
  \KOMAoptions{parskip=half}}
\makeatother
\usepackage{xcolor}
\setlength{\emergencystretch}{3em} % prevent overfull lines
\setcounter{secnumdepth}{-\maxdimen} % remove section numbering
% Make \paragraph and \subparagraph free-standing
\makeatletter
\ifx\paragraph\undefined\else
  \let\oldparagraph\paragraph
  \renewcommand{\paragraph}{
    \@ifstar
      \xxxParagraphStar
      \xxxParagraphNoStar
  }
  \newcommand{\xxxParagraphStar}[1]{\oldparagraph*{#1}\mbox{}}
  \newcommand{\xxxParagraphNoStar}[1]{\oldparagraph{#1}\mbox{}}
\fi
\ifx\subparagraph\undefined\else
  \let\oldsubparagraph\subparagraph
  \renewcommand{\subparagraph}{
    \@ifstar
      \xxxSubParagraphStar
      \xxxSubParagraphNoStar
  }
  \newcommand{\xxxSubParagraphStar}[1]{\oldsubparagraph*{#1}\mbox{}}
  \newcommand{\xxxSubParagraphNoStar}[1]{\oldsubparagraph{#1}\mbox{}}
\fi
\makeatother

\usepackage{color}
\usepackage{fancyvrb}
\newcommand{\VerbBar}{|}
\newcommand{\VERB}{\Verb[commandchars=\\\{\}]}
\DefineVerbatimEnvironment{Highlighting}{Verbatim}{commandchars=\\\{\}}
% Add ',fontsize=\small' for more characters per line
\usepackage{framed}
\definecolor{shadecolor}{RGB}{241,243,245}
\newenvironment{Shaded}{\begin{snugshade}}{\end{snugshade}}
\newcommand{\AlertTok}[1]{\textcolor[rgb]{0.68,0.00,0.00}{#1}}
\newcommand{\AnnotationTok}[1]{\textcolor[rgb]{0.37,0.37,0.37}{#1}}
\newcommand{\AttributeTok}[1]{\textcolor[rgb]{0.40,0.45,0.13}{#1}}
\newcommand{\BaseNTok}[1]{\textcolor[rgb]{0.68,0.00,0.00}{#1}}
\newcommand{\BuiltInTok}[1]{\textcolor[rgb]{0.00,0.23,0.31}{#1}}
\newcommand{\CharTok}[1]{\textcolor[rgb]{0.13,0.47,0.30}{#1}}
\newcommand{\CommentTok}[1]{\textcolor[rgb]{0.37,0.37,0.37}{#1}}
\newcommand{\CommentVarTok}[1]{\textcolor[rgb]{0.37,0.37,0.37}{\textit{#1}}}
\newcommand{\ConstantTok}[1]{\textcolor[rgb]{0.56,0.35,0.01}{#1}}
\newcommand{\ControlFlowTok}[1]{\textcolor[rgb]{0.00,0.23,0.31}{\textbf{#1}}}
\newcommand{\DataTypeTok}[1]{\textcolor[rgb]{0.68,0.00,0.00}{#1}}
\newcommand{\DecValTok}[1]{\textcolor[rgb]{0.68,0.00,0.00}{#1}}
\newcommand{\DocumentationTok}[1]{\textcolor[rgb]{0.37,0.37,0.37}{\textit{#1}}}
\newcommand{\ErrorTok}[1]{\textcolor[rgb]{0.68,0.00,0.00}{#1}}
\newcommand{\ExtensionTok}[1]{\textcolor[rgb]{0.00,0.23,0.31}{#1}}
\newcommand{\FloatTok}[1]{\textcolor[rgb]{0.68,0.00,0.00}{#1}}
\newcommand{\FunctionTok}[1]{\textcolor[rgb]{0.28,0.35,0.67}{#1}}
\newcommand{\ImportTok}[1]{\textcolor[rgb]{0.00,0.46,0.62}{#1}}
\newcommand{\InformationTok}[1]{\textcolor[rgb]{0.37,0.37,0.37}{#1}}
\newcommand{\KeywordTok}[1]{\textcolor[rgb]{0.00,0.23,0.31}{\textbf{#1}}}
\newcommand{\NormalTok}[1]{\textcolor[rgb]{0.00,0.23,0.31}{#1}}
\newcommand{\OperatorTok}[1]{\textcolor[rgb]{0.37,0.37,0.37}{#1}}
\newcommand{\OtherTok}[1]{\textcolor[rgb]{0.00,0.23,0.31}{#1}}
\newcommand{\PreprocessorTok}[1]{\textcolor[rgb]{0.68,0.00,0.00}{#1}}
\newcommand{\RegionMarkerTok}[1]{\textcolor[rgb]{0.00,0.23,0.31}{#1}}
\newcommand{\SpecialCharTok}[1]{\textcolor[rgb]{0.37,0.37,0.37}{#1}}
\newcommand{\SpecialStringTok}[1]{\textcolor[rgb]{0.13,0.47,0.30}{#1}}
\newcommand{\StringTok}[1]{\textcolor[rgb]{0.13,0.47,0.30}{#1}}
\newcommand{\VariableTok}[1]{\textcolor[rgb]{0.07,0.07,0.07}{#1}}
\newcommand{\VerbatimStringTok}[1]{\textcolor[rgb]{0.13,0.47,0.30}{#1}}
\newcommand{\WarningTok}[1]{\textcolor[rgb]{0.37,0.37,0.37}{\textit{#1}}}

\providecommand{\tightlist}{%
  \setlength{\itemsep}{0pt}\setlength{\parskip}{0pt}}\usepackage{longtable,booktabs,array}
\usepackage{calc} % for calculating minipage widths
% Correct order of tables after \paragraph or \subparagraph
\usepackage{etoolbox}
\makeatletter
\patchcmd\longtable{\par}{\if@noskipsec\mbox{}\fi\par}{}{}
\makeatother
% Allow footnotes in longtable head/foot
\IfFileExists{footnotehyper.sty}{\usepackage{footnotehyper}}{\usepackage{footnote}}
\makesavenoteenv{longtable}
\usepackage{graphicx}
\makeatletter
\def\maxwidth{\ifdim\Gin@nat@width>\linewidth\linewidth\else\Gin@nat@width\fi}
\def\maxheight{\ifdim\Gin@nat@height>\textheight\textheight\else\Gin@nat@height\fi}
\makeatother
% Scale images if necessary, so that they will not overflow the page
% margins by default, and it is still possible to overwrite the defaults
% using explicit options in \includegraphics[width, height, ...]{}
\setkeys{Gin}{width=\maxwidth,height=\maxheight,keepaspectratio}
% Set default figure placement to htbp
\makeatletter
\def\fps@figure{htbp}
\makeatother

\KOMAoption{captions}{tableheading}
\makeatletter
\@ifpackageloaded{tcolorbox}{}{\usepackage[skins,breakable]{tcolorbox}}
\@ifpackageloaded{fontawesome5}{}{\usepackage{fontawesome5}}
\definecolor{quarto-callout-color}{HTML}{909090}
\definecolor{quarto-callout-note-color}{HTML}{0758E5}
\definecolor{quarto-callout-important-color}{HTML}{CC1914}
\definecolor{quarto-callout-warning-color}{HTML}{EB9113}
\definecolor{quarto-callout-tip-color}{HTML}{00A047}
\definecolor{quarto-callout-caution-color}{HTML}{FC5300}
\definecolor{quarto-callout-color-frame}{HTML}{acacac}
\definecolor{quarto-callout-note-color-frame}{HTML}{4582ec}
\definecolor{quarto-callout-important-color-frame}{HTML}{d9534f}
\definecolor{quarto-callout-warning-color-frame}{HTML}{f0ad4e}
\definecolor{quarto-callout-tip-color-frame}{HTML}{02b875}
\definecolor{quarto-callout-caution-color-frame}{HTML}{fd7e14}
\makeatother
\makeatletter
\@ifpackageloaded{caption}{}{\usepackage{caption}}
\AtBeginDocument{%
\ifdefined\contentsname
  \renewcommand*\contentsname{Table of contents}
\else
  \newcommand\contentsname{Table of contents}
\fi
\ifdefined\listfigurename
  \renewcommand*\listfigurename{List of Figures}
\else
  \newcommand\listfigurename{List of Figures}
\fi
\ifdefined\listtablename
  \renewcommand*\listtablename{List of Tables}
\else
  \newcommand\listtablename{List of Tables}
\fi
\ifdefined\figurename
  \renewcommand*\figurename{Figure}
\else
  \newcommand\figurename{Figure}
\fi
\ifdefined\tablename
  \renewcommand*\tablename{Table}
\else
  \newcommand\tablename{Table}
\fi
}
\@ifpackageloaded{float}{}{\usepackage{float}}
\floatstyle{ruled}
\@ifundefined{c@chapter}{\newfloat{codelisting}{h}{lop}}{\newfloat{codelisting}{h}{lop}[chapter]}
\floatname{codelisting}{Listing}
\newcommand*\listoflistings{\listof{codelisting}{List of Listings}}
\makeatother
\makeatletter
\makeatother
\makeatletter
\@ifpackageloaded{caption}{}{\usepackage{caption}}
\@ifpackageloaded{subcaption}{}{\usepackage{subcaption}}
\makeatother

\ifLuaTeX
  \usepackage{selnolig}  % disable illegal ligatures
\fi
\usepackage{bookmark}

\IfFileExists{xurl.sty}{\usepackage{xurl}}{} % add URL line breaks if available
\urlstyle{same} % disable monospaced font for URLs
\hypersetup{
  pdftitle={Advanced Medical Statistics -- Lab 4},
  colorlinks=true,
  linkcolor={blue},
  filecolor={Maroon},
  citecolor={Blue},
  urlcolor={Blue},
  pdfcreator={LaTeX via pandoc}}


\title{Advanced Medical Statistics -- Lab 4}
\usepackage{etoolbox}
\makeatletter
\providecommand{\subtitle}[1]{% add subtitle to \maketitle
  \apptocmd{\@title}{\par {\large #1 \par}}{}{}
}
\makeatother
\subtitle{R version}
\author{}
\date{}

\begin{document}
\maketitle


Welcome to lab 4 in the advanced medical statistics course. In this lab,
we will focus on the analysis of categorical data and the comparison of
proportions between groups. We will also perform several statistical
tests for the analysis of paired data.

\section{Part 1: Inference for categorical
data}\label{part-1-inference-for-categorical-data}

\subsection{Smoking and post-surgical
complications}\label{smoking-and-post-surgical-complications}

A study was conducted to investigate whether smoking is associated with
an increased risk of post-surgical complications. The relationship
between smoking status (smoker or non-smoker) and the occurrence of
complications following surgery was examined. The outcome of interest
was whether or not a complication occurred (yes or no), with smoking
status serving as the explanatory variable to compare complication rates
between the two groups.

The data from the study are summarized in the following 2x2 contingency
table:

\begin{longtable}[]{@{}llll@{}}
\toprule\noalign{}
& Complication & No Complication & Total \\
\midrule\noalign{}
\endhead
\bottomrule\noalign{}
\endlastfoot
Smokers & 8 & 12 & 20 \\
Non-smokers & 10 & 50 & 60 \\
Total & 18 & 62 & 80 \\
\end{longtable}

\subsubsection{Confidence intervals and hypothesis testing for the
difference in proportions using the normal
approximation}\label{confidence-intervals-and-hypothesis-testing-for-the-difference-in-proportions-using-the-normal-approximation}

\begin{tcolorbox}[enhanced jigsaw, bottomrule=.15mm, coltitle=black, colbacktitle=quarto-callout-important-color!10!white, left=2mm, bottomtitle=1mm, breakable, colframe=quarto-callout-important-color-frame, toprule=.15mm, titlerule=0mm, title={Question 1}, opacitybacktitle=0.6, arc=.35mm, rightrule=.15mm, opacityback=0, leftrule=.75mm, toptitle=1mm, colback=white]

Using the data provided in the table, calculate an approximate 95\%
confidence interval for the difference in proportions of post-surgical
complications between smokers and non-smokers.

\end{tcolorbox}

\begin{tcolorbox}[enhanced jigsaw, bottomrule=.15mm, coltitle=black, colbacktitle=quarto-callout-important-color!10!white, left=2mm, bottomtitle=1mm, breakable, colframe=quarto-callout-important-color-frame, toprule=.15mm, titlerule=0mm, title={Question 2}, opacitybacktitle=0.6, arc=.35mm, rightrule=.15mm, opacityback=0, leftrule=.75mm, toptitle=1mm, colback=white]

Based on the 95\% confidence interval, can we conclude that there is a
statistically significant difference in the proportion of post-surgical
complications between smokers and non-smokers?

\end{tcolorbox}

We can also use the normal approximation to test the hypothesis that the
proportion of complications is the same for smokers and non-smokers.
This test is known as the two-sample Z test for equality of proportions.

As explained in the syllabus, the two-sample Z test uses the pooled
population proportion \(\hat{p}\), which is calculated as the total
number of events divided by the total sample size. This pooled
proportion is used under the null hypothesis, which assumes that the two
groups share the same underlying proportion. The standard error of the
difference in proportions is then calculated as
\(\sqrt{\hat{p}(1-\hat{p})(1/n_1 + 1/n_2)}\), where \(n_1\) and \(n_2\)
are the sample sizes in the two groups.

In contrast, the 95\% confidence interval for the difference in
proportions does not rely on the pooled proportion. Instead, it
calculates the standard error separately for each group using the
observed proportions, resulting in an unpooled standard error:
\(\sqrt{\frac{p_1(1-p_1)}{n_1} + \frac{p_2(1-p_2)}{n_2}}\), where
\(p_1\) and \(p_2\) are the sample proportions for each group. This
approach provides an interval that better reflects the variability in
the observed data, independent of the null hypothesis assumption.

To conduct the two-sample Z test, we start by creating a contingency
table:

\begin{Shaded}
\begin{Highlighting}[]
\CommentTok{\# Create a contingency table}
\NormalTok{complications }\OtherTok{\textless{}{-}} \FunctionTok{matrix}\NormalTok{(}\FunctionTok{c}\NormalTok{(}\DecValTok{8}\NormalTok{, }\DecValTok{12}\NormalTok{, }\DecValTok{10}\NormalTok{, }\DecValTok{50}\NormalTok{), }\AttributeTok{nrow =} \DecValTok{2}\NormalTok{, }\AttributeTok{byrow =} \ConstantTok{TRUE}\NormalTok{)}
\FunctionTok{colnames}\NormalTok{(complications) }\OtherTok{\textless{}{-}} \FunctionTok{c}\NormalTok{(}\StringTok{"Complication"}\NormalTok{, }\StringTok{"No Complication"}\NormalTok{)}
\FunctionTok{rownames}\NormalTok{(complications) }\OtherTok{\textless{}{-}} \FunctionTok{c}\NormalTok{(}\StringTok{"Smokers"}\NormalTok{, }\StringTok{"Non{-}smokers"}\NormalTok{)}
\NormalTok{complications }\OtherTok{\textless{}{-}} \FunctionTok{as.table}\NormalTok{(complications)}
\NormalTok{complications}
\end{Highlighting}
\end{Shaded}

Next, we supply this table to the \texttt{prop.test()} function to
calculate the test statistic and p-value:

\begin{Shaded}
\begin{Highlighting}[]
\FunctionTok{prop.test}\NormalTok{(complications)}
\end{Highlighting}
\end{Shaded}

\begin{tcolorbox}[enhanced jigsaw, bottomrule=.15mm, coltitle=black, colbacktitle=quarto-callout-important-color!10!white, left=2mm, bottomtitle=1mm, breakable, colframe=quarto-callout-important-color-frame, toprule=.15mm, titlerule=0mm, title={Question 3}, opacitybacktitle=0.6, arc=.35mm, rightrule=.15mm, opacityback=0, leftrule=.75mm, toptitle=1mm, colback=white]

Based on the results of the test, can we conclude that there is a
statistically significant difference in the proportion of post-surgical
complications between smokers and non-smokers?

\end{tcolorbox}

\begin{tcolorbox}[enhanced jigsaw, bottomrule=.15mm, coltitle=black, colbacktitle=quarto-callout-important-color!10!white, left=2mm, bottomtitle=1mm, breakable, colframe=quarto-callout-important-color-frame, toprule=.15mm, titlerule=0mm, title={Question 4}, opacitybacktitle=0.6, arc=.35mm, rightrule=.15mm, opacityback=0, leftrule=.75mm, toptitle=1mm, colback=white]

In addition to the p-value, output of the \texttt{prop.test()} function
also provides an approximate 95\% confidence interval for the difference
in proportions. How does this confidence interval compare to the one you
calculated manually?

\end{tcolorbox}

\subsubsection{Checking of assumptions}\label{checking-of-assumptions}

For the use of the normal approximation to be valid, the expected number
of events and non-events in each group should be at least 5.

\begin{tcolorbox}[enhanced jigsaw, bottomrule=.15mm, coltitle=black, colbacktitle=quarto-callout-important-color!10!white, left=2mm, bottomtitle=1mm, breakable, colframe=quarto-callout-important-color-frame, toprule=.15mm, titlerule=0mm, title={Exercise}, opacitybacktitle=0.6, arc=.35mm, rightrule=.15mm, opacityback=0, leftrule=.75mm, toptitle=1mm, colback=white]

Check this assumption by calculating the expected counts for each cell
in the contingency table.

\end{tcolorbox}

\begin{tcolorbox}[enhanced jigsaw, bottomrule=.15mm, coltitle=black, colbacktitle=quarto-callout-important-color!10!white, left=2mm, bottomtitle=1mm, breakable, colframe=quarto-callout-important-color-frame, toprule=.15mm, titlerule=0mm, title={Question 5}, opacitybacktitle=0.6, arc=.35mm, rightrule=.15mm, opacityback=0, leftrule=.75mm, toptitle=1mm, colback=white]

Is it reasonable to use the normal approximation in this case?

\end{tcolorbox}

\subsubsection{Fisher's exact test}\label{fishers-exact-test}

When the expected cell counts are small, the normal approximation may
not be appropriate. In such cases, Fisher's exact test is recommended
for testing the association between two categorical variables.

In R, we can perform Fisher's exact test using the
\texttt{fisher.test()} function. The test is based on the hypergeometric
distribution and provides an exact p-value for the association between
the two variables:

\begin{Shaded}
\begin{Highlighting}[]
\FunctionTok{fisher.test}\NormalTok{(complications)}
\end{Highlighting}
\end{Shaded}

\begin{tcolorbox}[enhanced jigsaw, bottomrule=.15mm, coltitle=black, colbacktitle=quarto-callout-important-color!10!white, left=2mm, bottomtitle=1mm, breakable, colframe=quarto-callout-important-color-frame, toprule=.15mm, titlerule=0mm, title={Question 6}, opacitybacktitle=0.6, arc=.35mm, rightrule=.15mm, opacityback=0, leftrule=.75mm, toptitle=1mm, colback=white]

Based on the results of Fisher's exact test, can we conclude that there
is a statistically significant difference in the proportion of
post-surgical complications between smokers and non-smokers?

\end{tcolorbox}

\subsection{Vaccine side effects across age
groups}\label{vaccine-side-effects-across-age-groups}

A study was conducted to investigate whether the occurrence of vaccine
side effects differs across age groups. Researchers categorized side
effects into three types: none, mild, and severe. The study participants
were divided into three age groups: 18--39, 40--59, and 60+, and data
was collected on the type of side effect experienced by individuals in
each group.

The research objective was to determine whether the distribution of side
effects is consistent across these age groups.

The data is summarized in the following contingency table:

\begin{longtable}[]{@{}lllll@{}}
\toprule\noalign{}
Age Group & None & Mild & Severe & Total \\
\midrule\noalign{}
\endhead
\bottomrule\noalign{}
\endlastfoot
18--39 & 50 & 30 & 10 & 90 \\
40--59 & 40 & 40 & 20 & 100 \\
60+ & 30 & 50 & 40 & 120 \\
Total & 120 & 120 & 70 & 310 \\
\end{longtable}

\subsubsection{Chi-square test of
homogeneity}\label{chi-square-test-of-homogeneity}

To conduct a chi-square test of homogeneity, we start by creating a
contingency table:

\begin{Shaded}
\begin{Highlighting}[]
\CommentTok{\# Create a contingency table}
\NormalTok{side\_effects }\OtherTok{\textless{}{-}} \FunctionTok{matrix}\NormalTok{(}\FunctionTok{c}\NormalTok{(}\DecValTok{50}\NormalTok{, }\DecValTok{30}\NormalTok{, }\DecValTok{10}\NormalTok{, }\DecValTok{40}\NormalTok{, }\DecValTok{40}\NormalTok{, }\DecValTok{20}\NormalTok{, }\DecValTok{30}\NormalTok{, }\DecValTok{50}\NormalTok{, }\DecValTok{40}\NormalTok{), }\AttributeTok{nrow =} \DecValTok{3}\NormalTok{, }\AttributeTok{byrow =} \ConstantTok{TRUE}\NormalTok{)}
\FunctionTok{colnames}\NormalTok{(side\_effects) }\OtherTok{\textless{}{-}} \FunctionTok{c}\NormalTok{(}\StringTok{"None"}\NormalTok{, }\StringTok{"Mild"}\NormalTok{, }\StringTok{"Severe"}\NormalTok{)}
\FunctionTok{rownames}\NormalTok{(side\_effects) }\OtherTok{\textless{}{-}} \FunctionTok{c}\NormalTok{(}\StringTok{"18–39"}\NormalTok{, }\StringTok{"40–59"}\NormalTok{, }\StringTok{"60+"}\NormalTok{)}
\NormalTok{side\_effects }\OtherTok{\textless{}{-}} \FunctionTok{as.table}\NormalTok{(side\_effects)}
\NormalTok{side\_effects}
\end{Highlighting}
\end{Shaded}

Next, we supply this table to the \texttt{chisq.test()} function to
calculate the test statistic and p-value:

\begin{Shaded}
\begin{Highlighting}[]
\NormalTok{chisq\_test\_overall }\OtherTok{\textless{}{-}} \FunctionTok{chisq.test}\NormalTok{(side\_effects)}
\FunctionTok{print}\NormalTok{(chisq\_test\_overall)}
\end{Highlighting}
\end{Shaded}

\begin{tcolorbox}[enhanced jigsaw, bottomrule=.15mm, coltitle=black, colbacktitle=quarto-callout-important-color!10!white, left=2mm, bottomtitle=1mm, breakable, colframe=quarto-callout-important-color-frame, toprule=.15mm, titlerule=0mm, title={Question 7}, opacitybacktitle=0.6, arc=.35mm, rightrule=.15mm, opacityback=0, leftrule=.75mm, toptitle=1mm, colback=white]

Based on the results of the chi-square test, can we conclude that the
distribution of vaccine side effects is consistent across the three age
groups?

\end{tcolorbox}

\paragraph{Checking of assumptions}\label{checking-of-assumptions-1}

To use the chi-square test, the expected cell counts should be at least
5 for most cells. To check this assumption, we retrieve the table of
expected counts from the output of the \texttt{chisq.test()} function,
which we stored in the \texttt{chisq\_test\_overall} object:

\begin{Shaded}
\begin{Highlighting}[]
\CommentTok{\# Retrieve the table of expected counts}
\NormalTok{chisq\_test\_overall}\SpecialCharTok{$}\NormalTok{expected}
\end{Highlighting}
\end{Shaded}

\begin{tcolorbox}[enhanced jigsaw, bottomrule=.15mm, coltitle=black, colbacktitle=quarto-callout-important-color!10!white, left=2mm, bottomtitle=1mm, breakable, colframe=quarto-callout-important-color-frame, toprule=.15mm, titlerule=0mm, title={Question 8}, opacitybacktitle=0.6, arc=.35mm, rightrule=.15mm, opacityback=0, leftrule=.75mm, toptitle=1mm, colback=white]

Are the expected cell counts greater than 5 for the different cells in
the contingency table?

\end{tcolorbox}

\paragraph{Post-hoc pairwise
comparisons}\label{post-hoc-pairwise-comparisons}

Finally, we are interested in determining which age groups have
significantly different distributions of side effects. For a
\(n \times 2\) table, this can be achieved easily using the
\texttt{pairwise.prop.test()} function, which performs pairwise
comparisons of proportions between groups with adjustments for multiple
testing. However, in our case, we are working with a \(3 \times 3\)
contingency table, where the outcomes have more than two categories
(None, Mild, Severe). This complexity requires a manual approach to
perform pairwise comparisons.

We start by comparing the first two age groups (18--39 and 40--59).
First, we set up the contingency table for these two groups:

\begin{Shaded}
\begin{Highlighting}[]
\NormalTok{table\_12 }\OtherTok{\textless{}{-}}\NormalTok{ side\_effects[}\FunctionTok{c}\NormalTok{(}\StringTok{"18–39"}\NormalTok{, }\StringTok{"40–59"}\NormalTok{), ]}
\NormalTok{table\_12}
\end{Highlighting}
\end{Shaded}

Next, we perform the chi-square test for this subset of the data and
adjust the p-value by applying the Bonferroni correction:

\begin{Shaded}
\begin{Highlighting}[]
\CommentTok{\# Perform chi{-}square test for the subset of data}
\NormalTok{chisq\_test\_12 }\OtherTok{\textless{}{-}} \FunctionTok{chisq.test}\NormalTok{(table\_12)}
\FunctionTok{print}\NormalTok{(chisq\_test\_12)}

\CommentTok{\# Adjust the p{-}value for multiple testing}
\NormalTok{p\_adjusted\_12 }\OtherTok{\textless{}{-}} \DecValTok{3}\SpecialCharTok{*}\NormalTok{chisq\_test\_12}\SpecialCharTok{$}\NormalTok{p.value}
\NormalTok{p\_adjusted\_12}
\end{Highlighting}
\end{Shaded}

\begin{tcolorbox}[enhanced jigsaw, bottomrule=.15mm, coltitle=black, colbacktitle=quarto-callout-important-color!10!white, left=2mm, bottomtitle=1mm, breakable, colframe=quarto-callout-important-color-frame, toprule=.15mm, titlerule=0mm, title={Exercise}, opacitybacktitle=0.6, arc=.35mm, rightrule=.15mm, opacityback=0, leftrule=.75mm, toptitle=1mm, colback=white]

Perform the pairwise comparison between the other two pairs of age
groups (40--59 and 60+, 18--39 and 60+) using the same approach.

\end{tcolorbox}

\begin{tcolorbox}[enhanced jigsaw, bottomrule=.15mm, coltitle=black, colbacktitle=quarto-callout-important-color!10!white, left=2mm, bottomtitle=1mm, breakable, colframe=quarto-callout-important-color-frame, toprule=.15mm, titlerule=0mm, title={Question 9}, opacitybacktitle=0.6, arc=.35mm, rightrule=.15mm, opacityback=0, leftrule=.75mm, toptitle=1mm, colback=white]

Based on the results of the pairwise comparisons, which age groups have
significantly different distributions of side effects?

\end{tcolorbox}

\section{Part 2: Analysis of paired continous
data}\label{part-2-analysis-of-paired-continous-data}

\subsection{Introduction}\label{introduction}

In this part of the lab, we will analyze paired data on pocket depth
before and after an intervention. Pocket depth refers to the depth of
the gum pockets around teeth, measured using a periodontal probe. It is
an important indicator of periodontal health. Healthy gums typically
have pocket depths less than 3 mm, while deeper pockets may indicate
conditions such as gingivitis or periodontitis.

The dataset \texttt{pockets\_paired.sav}, available on Brightspace,
contains the following columns:

\begin{itemize}
\tightlist
\item
  \texttt{subjectID}: Unique identifier for each participant
\item
  \texttt{pocket\_depth\_before}: Average pocket depth (in mm) measured
  before the intervention
\item
  \texttt{pocket\_depth\_after}: Average pocket depth (in mm) measured
  after the intervention
\end{itemize}

The objective is to determine whether the intervention significantly
reduces pocket depth. We will apply three statistical methods to analyze
the paired data:

\begin{enumerate}
\def\labelenumi{\arabic{enumi}.}
\tightlist
\item
  Paired t-test
\item
  Sign test
\item
  Wilcoxon signed-rank test
\end{enumerate}

\subsubsection{Paired t-test}\label{paired-t-test}

First, we load the data from the provided SPSS file
\texttt{pockets\_paired.sav}:

\begin{Shaded}
\begin{Highlighting}[]
\FunctionTok{library}\NormalTok{(haven)}
\NormalTok{pockets }\OtherTok{\textless{}{-}} \FunctionTok{read\_sav}\NormalTok{(}\StringTok{"pockets\_paired.sav"}\NormalTok{)}
\FunctionTok{head}\NormalTok{(pockets)}
\end{Highlighting}
\end{Shaded}

Next, we perform a paired t-test to compare the average pocket depth
before and after the intervention:

\begin{Shaded}
\begin{Highlighting}[]
\CommentTok{\# Perform paired t{-}test}
\NormalTok{t\_test }\OtherTok{\textless{}{-}} \FunctionTok{t.test}\NormalTok{(pockets}\SpecialCharTok{$}\NormalTok{pocket\_depth\_before, pockets}\SpecialCharTok{$}\NormalTok{pocket\_depth\_after, }\AttributeTok{paired =} \ConstantTok{TRUE}\NormalTok{)}
\FunctionTok{print}\NormalTok{(t\_test)}
\end{Highlighting}
\end{Shaded}

The \texttt{t.test()} function with the argument
\texttt{paired\ =\ TRUE} conducts a paired t-test; setting
\texttt{paired\ =\ FALSE} (default) would perform an independent samples
t-test. The output includes the test statistic, degrees of freedom, and
the p-value.

\begin{tcolorbox}[enhanced jigsaw, bottomrule=.15mm, coltitle=black, colbacktitle=quarto-callout-important-color!10!white, left=2mm, bottomtitle=1mm, breakable, colframe=quarto-callout-important-color-frame, toprule=.15mm, titlerule=0mm, title={Question 10}, opacitybacktitle=0.6, arc=.35mm, rightrule=.15mm, opacityback=0, leftrule=.75mm, toptitle=1mm, colback=white]

Based on the results of the paired t-test, can we conclude that the
intervention significantly reduces pocket depth?

\end{tcolorbox}

\paragraph{Checking of assumptions}\label{checking-of-assumptions-2}

To determine whether it is appropriate to apply the paired t-test to
these data, we need to verify that the differences in pocket depth
before and after the intervention are normally distributed. We can
visually inspect the distribution of differences using a histogram:

\begin{Shaded}
\begin{Highlighting}[]
\CommentTok{\# Calculate the differences in pocket depth}
\NormalTok{pockets}\SpecialCharTok{$}\NormalTok{diff }\OtherTok{\textless{}{-}}\NormalTok{ pockets}\SpecialCharTok{$}\NormalTok{pocket\_depth\_after }\SpecialCharTok{{-}}\NormalTok{ pockets}\SpecialCharTok{$}\NormalTok{pocket\_depth\_before}

\CommentTok{\# Create a histogram of the differences}
\FunctionTok{library}\NormalTok{(ggplot2)}
\FunctionTok{ggplot}\NormalTok{(pockets, }\FunctionTok{aes}\NormalTok{(}\AttributeTok{x =}\NormalTok{ diff)) }\SpecialCharTok{+}
  \FunctionTok{geom\_histogram}\NormalTok{(}\AttributeTok{binwidth =} \FloatTok{0.05}\NormalTok{, }\AttributeTok{fill =} \StringTok{"skyblue"}\NormalTok{, }\AttributeTok{color =} \StringTok{"black"}\NormalTok{) }\SpecialCharTok{+}
  \FunctionTok{labs}\NormalTok{(}\AttributeTok{title =} \StringTok{"Distribution of differences in pocket depth"}\NormalTok{,}
       \AttributeTok{x =} \StringTok{"Difference in pocket depth (mm)"}\NormalTok{,}
       \AttributeTok{y =} \StringTok{"Frequency"}\NormalTok{)}
\end{Highlighting}
\end{Shaded}

\begin{tcolorbox}[enhanced jigsaw, bottomrule=.15mm, coltitle=black, colbacktitle=quarto-callout-important-color!10!white, left=2mm, bottomtitle=1mm, breakable, colframe=quarto-callout-important-color-frame, toprule=.15mm, titlerule=0mm, title={Question 11}, opacitybacktitle=0.6, arc=.35mm, rightrule=.15mm, opacityback=0, leftrule=.75mm, toptitle=1mm, colback=white]

Based on the histogram, do the differences in pocket depth appear to be
approximately normally distributed?

\end{tcolorbox}

\subsubsection{Sign test}\label{sign-test}

The sign test is a non-parametric test used to compare two related
samples. It is based on the signs of the differences between the pairs
of observations. We will apply the sign test to the pocket depth data to
determine whether the intervention has a significant effect.

First, we calculate the total number of positive and negative signs in
the differences:

\begin{Shaded}
\begin{Highlighting}[]
\CommentTok{\# Calculate the number of positive and negative signs}
\NormalTok{signs\_positive }\OtherTok{\textless{}{-}} \FunctionTok{sum}\NormalTok{(pockets}\SpecialCharTok{$}\NormalTok{diff }\SpecialCharTok{\textgreater{}} \DecValTok{0}\NormalTok{)}
\NormalTok{signs\_negative }\OtherTok{\textless{}{-}} \FunctionTok{sum}\NormalTok{(pockets}\SpecialCharTok{$}\NormalTok{diff }\SpecialCharTok{\textless{}} \DecValTok{0}\NormalTok{)}
\end{Highlighting}
\end{Shaded}

Next, we perform the sign test using the \texttt{binom.test()} function,
which calculates the exact p-value for the sign test:

\begin{Shaded}
\begin{Highlighting}[]
\CommentTok{\# Perform the sign test}
\CommentTok{\# There are 7 positive signs out of 96 pairs with either a positive or negative sign}
\NormalTok{sign\_test }\OtherTok{\textless{}{-}} \FunctionTok{binom.test}\NormalTok{(signs\_positive, (signs\_positive }\SpecialCharTok{+}\NormalTok{ signs\_negative), }\AttributeTok{p =} \FloatTok{0.5}\NormalTok{)}
\FunctionTok{print}\NormalTok{(sign\_test)}
\end{Highlighting}
\end{Shaded}

\begin{tcolorbox}[enhanced jigsaw, bottomrule=.15mm, coltitle=black, colbacktitle=quarto-callout-important-color!10!white, left=2mm, bottomtitle=1mm, breakable, colframe=quarto-callout-important-color-frame, toprule=.15mm, titlerule=0mm, title={Question 12}, opacitybacktitle=0.6, arc=.35mm, rightrule=.15mm, opacityback=0, leftrule=.75mm, toptitle=1mm, colback=white]

Based on the results of the sign test, can we conclude that the
intervention significantly reduces pocket depth?

\end{tcolorbox}

\subsubsection{Wilcoxon signed-rank
test}\label{wilcoxon-signed-rank-test}

The Wilcoxon signed-rank test is another non-parametric test used to
compare two related samples. It is based on the ranks of the absolute
differences between the pairs of observations. In this case, the sign
test is more appropriate because the Wilcoxon signed-rank test requires
the assumption of symmetry in the distribution of differences, whereas
the previously constructed histogram suggests that the distribution of
these differences is left-skewed.

To perform the Wilcoxon signed-rank test, we use the
\texttt{wilcox.test()} function:

\begin{Shaded}
\begin{Highlighting}[]
\CommentTok{\# Perform the Wilcoxon signed{-}rank test}
\NormalTok{wilcox\_test }\OtherTok{\textless{}{-}} \FunctionTok{wilcox.test}\NormalTok{(pockets}\SpecialCharTok{$}\NormalTok{pocket\_depth\_before, pockets}\SpecialCharTok{$}\NormalTok{pocket\_depth\_after, }\AttributeTok{paired =} \ConstantTok{TRUE}\NormalTok{)}
\FunctionTok{print}\NormalTok{(wilcox\_test)}
\end{Highlighting}
\end{Shaded}

\begin{tcolorbox}[enhanced jigsaw, bottomrule=.15mm, coltitle=black, colbacktitle=quarto-callout-important-color!10!white, left=2mm, bottomtitle=1mm, breakable, colframe=quarto-callout-important-color-frame, toprule=.15mm, titlerule=0mm, title={Question 13}, opacitybacktitle=0.6, arc=.35mm, rightrule=.15mm, opacityback=0, leftrule=.75mm, toptitle=1mm, colback=white]

Based on the results of the Wilcoxon signed-rank test, can we conclude
that the intervention significantly reduces pocket depth?

\end{tcolorbox}

\section{Part 3: Analysis of paired dichotomous
data}\label{part-3-analysis-of-paired-dichotomous-data}

In this part of the lab, we will analyze paired dichotomous data from a
study investigating the skin response to two substances:
\textbf{dinitrochlorobenzene (DNCB)}, a contact allergen, and
\textbf{croton oil}, a skin irritant. The objective is to determine
whether the proportion of patients with a negative response to DNCB is
the same as the proportion with a negative response to croton oil.

The data, summarized in the following table, represents the results of
simultaneous skin reaction tests on 173 patients with skin cancer:

\begin{longtable}[]{@{}llll@{}}
\toprule\noalign{}
& DNCB +ve & DNCB -ve & Total \\
\midrule\noalign{}
\endhead
\bottomrule\noalign{}
\endlastfoot
\textbf{Croton oil +ve} & 81 & 23 & 104 \\
\textbf{Croton oil -ve} & 48 & 21 & 69 \\
\textbf{Total} & 129 & 44 & 173 \\
\end{longtable}

Because the data are paired, we will use the McNemar test to compare the
proportions of positive and negative responses to DNCB and croton oil.

First, we create a 2x2 contingency table from the data:

\begin{Shaded}
\begin{Highlighting}[]
\CommentTok{\# Create the table for DNCB and Croton Oil responses}
\NormalTok{skin\_response\_table }\OtherTok{\textless{}{-}} \FunctionTok{matrix}\NormalTok{(}\FunctionTok{c}\NormalTok{(}\DecValTok{81}\NormalTok{, }\DecValTok{23}\NormalTok{, }\DecValTok{48}\NormalTok{, }\DecValTok{21}\NormalTok{), }\AttributeTok{nrow =} \DecValTok{2}\NormalTok{, }\AttributeTok{byrow =} \ConstantTok{TRUE}\NormalTok{)}
\FunctionTok{colnames}\NormalTok{(skin\_response\_table) }\OtherTok{\textless{}{-}} \FunctionTok{c}\NormalTok{(}\StringTok{"DNCB +ve"}\NormalTok{, }\StringTok{"DNCB {-}ve"}\NormalTok{)}
\FunctionTok{rownames}\NormalTok{(skin\_response\_table) }\OtherTok{\textless{}{-}} \FunctionTok{c}\NormalTok{(}\StringTok{"Croton Oil +ve"}\NormalTok{, }\StringTok{"Croton Oil {-}ve"}\NormalTok{)}
\NormalTok{skin\_response\_table }\OtherTok{\textless{}{-}} \FunctionTok{as.table}\NormalTok{(skin\_response\_table)}

\CommentTok{\# Print the table}
\NormalTok{skin\_response\_table}
\end{Highlighting}
\end{Shaded}

Next, we perform the McNemar test using the \texttt{mcnemar.test()}
function:

\begin{Shaded}
\begin{Highlighting}[]
\CommentTok{\# Perform the McNemar test}
\NormalTok{mcnemar\_test }\OtherTok{\textless{}{-}} \FunctionTok{mcnemar.test}\NormalTok{(skin\_response\_table)}
\FunctionTok{print}\NormalTok{(mcnemar\_test)}
\end{Highlighting}
\end{Shaded}

\begin{tcolorbox}[enhanced jigsaw, bottomrule=.15mm, coltitle=black, colbacktitle=quarto-callout-important-color!10!white, left=2mm, bottomtitle=1mm, breakable, colframe=quarto-callout-important-color-frame, toprule=.15mm, titlerule=0mm, title={Question 14}, opacitybacktitle=0.6, arc=.35mm, rightrule=.15mm, opacityback=0, leftrule=.75mm, toptitle=1mm, colback=white]

Based on the results of the McNemar test, can we conclude that there is
a significant difference in the proportions of patients with a negative
response to DNCB and croton oil? If so, can you determine which
substance is associated with a higher proportion of negative responses?

\end{tcolorbox}




\end{document}
